\chapter{Validation and Benchmarking: Experimental Setup and Codes} \label{ap:validation}

To perform the validation of our technique, tests were run on a workstation with the following configuration:

\begin{itemize}
    \item CPU - AMD Ryzen Threadripper 3960X - 24 Cores with 2-way SMT, clocked at 2.2GHz, boosting to 4.5GHz
    \item RAM - 2x16 GiB at 3200 MT/s, in dual channel
    \item OS - Ubuntu 18.04.6 LTS
    \item Java - Azul Systems Zulu OpenJDK 11.0.5
    \item Compilers - clang v6.0.0 (experiment B), CompCert v3.10 (Coq platform v2022.04.1) (experiment B), gcc v7.5.0 (experiments B and C)
    \item Clava - built from source, commit 275e87b
    \item LARA Framework - built from source, commit 81bf5b62
    \item SPeCS Java Libraries - built from source, commit fa827db5
\end{itemize}

To build Clava from source, the repositories for Clava, the \href{https://github.com/specs-feup/lara-framework/tree/81bf5b624b0aa37ddd0bebda2b24e2a126f12609}{LARA Framework} and the \href{https://github.com/specs-feup/specs-java-libs/tree/fa827db586d7b2cbf94fff98873ee339f04cb694}{SPeCS Java Libraries} were cloned and imported into the Eclipse IDE, according to the instructions in the \href{https://github.com/specs-feup/specs-java-libs/blob/fa827db586d7b2cbf94fff98873ee339f04cb694/README.md}{SPeCS Java Libraries README file}, and the build target ClavaWeaver -> Java Application -> ClavaWeaverLauncher -> ClavaWeaverGUI was used to run the Clava toolchain.

\section{Experiment A}

\subsection{C Source Codes}

\subsubsection{K-Means}

Randomly distributes points around a set of pre-defined seed coordinates, and runs k-means to find clusters of points.

\begin{lstlisting}[caption=k\_means.c]
#include <stdio.h>
#include <stdlib.h>
#include <time.h>

float square_distance(float x1, float y1, float x2, float y2)
{
    float x_diff = (x1 - x2) * (x1 - x2);
    float y_diff = (y1 - y2) * (y1 - y2);
    float result = x_diff + y_diff;
    return result;
}

void calculate_centroid_means(size_t n_clusters, size_t gen_factor, float *centroid_xs, float *centroid_ys, size_t *centroid_counts, float *xs, float *ys, size_t *classes)
{
    // reset centroid means
    for (size_t k = 0; k < n_clusters; k++)
    {
        centroid_xs[k] = 0.0;
        centroid_ys[k] = 0.0;
        centroid_counts[k] = 0;
    }

    // update centroid means
    for (size_t i = 0; i < n_clusters * gen_factor; i++)
    {
        float x = xs[i];
        float y = ys[i];
        size_t k = classes[i];

        centroid_counts[k]++;
        centroid_xs[k] += (x - centroid_xs[k]) / centroid_counts[k];
        centroid_ys[k] += (y - centroid_ys[k]) / centroid_counts[k];
    }
}

int main()
{
    size_t n_clusters = 8;
    float cluster_seed_xs[8];
    cluster_seed_xs[0] = 3.0;
    cluster_seed_xs[1] = -2.1;
    cluster_seed_xs[2] = 12.8;
    cluster_seed_xs[3] = 0.3;
    cluster_seed_xs[4] = -7.5;
    cluster_seed_xs[5] = 16.2;
    cluster_seed_xs[6] = -13.1;
    cluster_seed_xs[7] = -0.7;
    float cluster_seed_ys[8];
    cluster_seed_ys[0] = 3.0;
    cluster_seed_ys[1] = 2.1;
    cluster_seed_ys[2] = -12.8;
    cluster_seed_ys[3] = -0.3;
    cluster_seed_ys[4] = 7.5;
    cluster_seed_ys[5] = 16.2;
    cluster_seed_ys[6] = -13.1;
    cluster_seed_ys[7] = -0.7;

    size_t allowed_switches = 2;
    size_t gen_factor = 150000;
    int dist_factor = 300;

    srand(42);

    float *xs = calloc(n_clusters * gen_factor, sizeof(float));
    float *ys = calloc(n_clusters * gen_factor, sizeof(float));
    size_t *classes = calloc(n_clusters * gen_factor, sizeof(size_t));

    // randomly distribute points around seed clusters
    for (size_t k = 0; k < n_clusters; k++)
    {
        size_t base = k * gen_factor;
        float cluster_x = cluster_seed_xs[k];
        float cluster_y = cluster_seed_ys[k];
        for (size_t offset = 0; offset < gen_factor; offset++)
        {
            size_t ix = base + offset;
            float x_delta = (rand() % (dist_factor * 2) - dist_factor) / (float)100;
            xs[ix] = cluster_x + x_delta;
            float y_delta = (rand() % (dist_factor * 2) - dist_factor) / (float)100;
            ys[ix] = cluster_y + y_delta;
            classes[ix] = rand() % n_clusters;
        }
    }

    // randomly distribute classes
    for (size_t i = 0; i < n_clusters * gen_factor; i++)
    {
        int class = rand() % n_clusters;
        classes[i] = class;
    }

    float centroid_xs[8];
    float centroid_ys[8];
    size_t centroid_counts[8];

    size_t switched_observations = 0;
    int iterations = 0;

    clock_t start, end;
    start = clock();
    do
    {
        switched_observations = 0;
        iterations++;
        calculate_centroid_means(n_clusters, gen_factor, centroid_xs, centroid_ys, centroid_counts, xs, ys, classes);
        for (size_t i = 0; i < n_clusters * gen_factor; i++)
        {
            float x = xs[i];
            float y = ys[i];
            size_t class = classes[i];

            for (size_t k = 0; k < n_clusters; k++)
            {
                float centroid_x = centroid_xs[k];
                float centroid_y = centroid_ys[k];
                float centroid_distance = square_distance(x, y, centroid_x, centroid_y);

                float class_x = centroid_xs[class];
                float class_y = centroid_ys[class];
                float class_distance = square_distance(x, y, class_x, class_y);

                if (centroid_distance < class_distance)
                {
                    class = k;
                }

                if (class != classes[i])
                {
                    classes[i] = class;
                    switched_observations++;
                }
            }
        }
    } while (switched_observations > allowed_switches);
    end = clock();
    fprintf(stderr, "%d\n", iterations);
    printf("%f\n", (end - start) / (double)(CLOCKS_PER_SEC / 1000));
}
\end{lstlisting}

\subsubsection{Matrix Multiplication and Trace}

Pre-populates two square matrices of similar rank, multiplies them, and gives the trace of the resulting matrix.

\begin{lstlisting}[caption=matrix\_mult.c]
#include <stdio.h>
#include <stdlib.h>
#include <time.h>
#include <stdbool.h>

typedef struct matrix
{
    size_t side;
    double **rows;
} matrix;

matrix *matrix_new(size_t sqrt_n)
{
    size_t side = sqrt_n * sqrt_n;
    double **rows = calloc(side, sizeof(double *));

    for (size_t row = 0; row < side; row++)
    {
        double *r = calloc(side, sizeof(double));
        rows[row] = r;
    }

    matrix *out = (matrix *)malloc(sizeof(matrix));
    out->side = side;
    out->rows = rows;

    return out;
}

void matrix_destroy(matrix *m)
{
    double **rows = m->rows;

    for (size_t row = 0; row < m->side; row++)
    {
        free(rows[row]);
    }

    free(rows);
    free(m);
}

double matrix_mult_coord(matrix *a, matrix *b, size_t i, size_t j)
{
    double acc = 0.0;

    for (size_t k = 0; k < a->side; k++)
    {
        acc += a->rows[i][k] * b->rows[k][j];
    }

    return acc;
}

void matrix_mult(matrix *a, matrix *b, matrix *out)
{
    size_t side = a->side;

    for (size_t i = 0; i < side; i++)
    {
        for (size_t j = 0; j < side; j++)
        {
            out->rows[i][j] = matrix_mult_coord(a, b, i, j);
        }
    }
}

void matrix_fill_rand(matrix *m)
{
    for (size_t i = 0; i < m->side; i++)
    {
        for (size_t j = 0; j < m->side; j++)
        {
            m->rows[i][j] = ((double)rand()) / 100.0;
        }
    }
}

double matrix_trace(matrix *m)
{
    double trace = 0.0;

    for (size_t i = 0; i < m->side; i++)
    {
        trace += m->rows[i][i];
    }

    return trace;
}

int main(void)
{
    srand(42);

    matrix *m1 = matrix_new(35);
    matrix *m2 = matrix_new(35);
    matrix *out = matrix_new(35);

    matrix_fill_rand(m1);
    matrix_fill_rand(m2);

    clock_t start = clock();
    matrix_mult(m1, m2, out);
    clock_t end = clock();
    printf("%f\n", (end - start) / (double)(CLOCKS_PER_SEC / 1000));
    double trace = matrix_trace(out);
    fprintf(stderr, "%f\n", trace);

    return EXIT_SUCCESS;
}
\end{lstlisting}

\subsection{Array Slice Aggregations}

Pre-populates a fixed array with random non-negative integers, and aggregates them.

\begin{lstlisting}[caption=vec\_multiple\_returns.c]
#include <stdio.h>
#include <stdlib.h>
#include <time.h>

int *slice_new(size_t size)
{
    return (int *)calloc(size, sizeof(int));
}

void slice_rand_fill(int *slice, size_t size)
{
    for (size_t i = 0; i < size; i++)
    {
        slice[i] = rand();
    }
}

int *slice_ref(int *slice, size_t size, size_t offset)
{
    if (offset >= size)
    {
        return NULL;
    }
    return slice + offset;
}

size_t slice_size(int *slice, size_t size)
{
    return size;
}

int main()
{
    srand(42);

    size_t size = 1000000000;
    int *slice = slice_new(size);
    slice_rand_fill(slice, size);

    unsigned int acc = 0;

    clock_t start = clock();
    for (size_t i = 0; i < slice_size(slice, size); i++)
    {
        int *elem;
        elem = slice_ref(slice, size, i);
        acc += (unsigned int)*elem;
        acc *= (unsigned int)*elem;
        acc += (unsigned int)*elem;
        acc *= (unsigned int)*elem;
        acc *= acc;
        acc += (unsigned int)*elem;
        acc *= (unsigned int)*elem;
        acc += (unsigned int)*elem;
        acc *= (unsigned int)*elem;
    }
    clock_t end = clock();
    fprintf(stderr, "%u\n", acc);

    printf("%f\n", (end - start) / (double)(CLOCKS_PER_SEC / 1000));
    return EXIT_SUCCESS;
}
\end{lstlisting}

\subsection{Clava Scripts}

These Clava scripts try to inline all possible calls in a given set of source-files, determined by the Clava config file, and report statistics about the inlinings.

\begin{lstlisting}[language=js,caption=Clava built-in inlining with no normalization]
laraImport("clava.code.Inliner");
laraImport("weaver.Query");

let calls_total = 0;
let calls_no_definition = 0;
let calls_inlined = 0;
let calls_inlining_fails = 0;

const inliner = new Inliner();

for (const call of Query.search("call")) {
  calls_total += 1;
  if (!call.function.isImplementation) {
    calls_no_definition += 1;
    continue;
  }
  try {
    println(
      `Trying to inline call '${call.code}' at location ${call.location}`
    );
    call.inline();
    println("Inlined successfully\n");
    calls_inlined += 1;
  } catch (e) {
    println(`Inlining failed: ${e.message}\n`);
    calls_inlining_fails += 1;
  }
}

println("Results:");
println(
  JSON.stringify(
    { calls_total, calls_no_definition, calls_inlined, calls_inlining_fails },
    undefined,
    2
  )
);
\end{lstlisting}

\begin{lstlisting}[language=js,caption=Clava built-in inlining with normalization]
laraImport("clava.code.Inliner");
laraImport("weaver.Query");
laraImport("clava.opt.NormalizeToSubset");
laraImport("clava.opt.PrepareForInlining");

let calls_total = 0;
let calls_no_definition = 0;
let calls_inlined = 0;
let calls_inlining_fails = 0;

NormalizeToSubset();

for (const call of Query.search("call")) {
  calls_total += 1;
  if (!call.function.isImplementation) {
    calls_no_definition += 1;
    continue;
  }
  try {
    println(
      `Trying to inline call '${call.code}' at location ${call.location}`
    );
    PrepareForInlining(call.function);
    call.inline();
    println("Inlined successfully\n");
    calls_inlined += 1;
  } catch (e) {
    println(`Inlining failed: ${e.message}\n`);
    calls_inlining_fails += 1;
  }
}

println("Results:");
println(
  JSON.stringify(
    { calls_total, calls_no_definition, calls_inlined, calls_inlining_fails },
    undefined,
    2
  )
);
\end{lstlisting}

\begin{lstlisting}[language=js,caption=Newly developed inlining with no normalization]
laraImport("clava.code.Inliner");
laraImport("weaver.Query");

let calls_total = 0;
let calls_no_definition = 0;
let calls_inlined = 0;
let calls_inlining_fails = 0;

const inliner = new Inliner();

for (const call of Query.search("call")) {
  calls_total += 1;
  if (!call.function.isImplementation) {
    calls_no_definition += 1;
    continue;
  }
  try {
    println(
      `Trying to inline call '${call.code}' at location ${call.location}`
    );
    const exprStmt = call.ancestor("exprStmt");
    inliner.inline(exprStmt);
    println("Inlined successfully\n");
    calls_inlined += 1;
  } catch (e) {
    println(`Inlining failed: ${e.message}\n`);
    calls_inlining_fails += 1;
  }
}

println("Results:");
println(
  JSON.stringify(
    { calls_total, calls_no_definition, calls_inlined, calls_inlining_fails },
    undefined,
    2
  )
);
\end{lstlisting}

\begin{lstlisting}[language=js,caption=Newly developed inlining with normalization]
laraImport("clava.code.Inliner");
laraImport("weaver.Query");
laraImport("clava.opt.NormalizeToSubset");
laraImport("clava.opt.PrepareForInlining");

let calls_total = 0;
let calls_no_definition = 0;
let calls_inlined = 0;
let calls_inlining_fails = 0;

NormalizeToSubset();

const inliner = new Inliner();

for (const call of Query.search("call")) {
  calls_total += 1;
  if (!call.function.isImplementation) {
    calls_no_definition += 1;
    continue;
  }
  try {
    println(
      `Trying to inline call '${call.code}' at location ${call.location}`
    );
    PrepareForInlining(call.function);
    const exprStmt = call.ancestor("exprStmt");
    inliner.inline(exprStmt);
    println("Inlined successfully\n");
    calls_inlined += 1;
  } catch (e) {
    println(`Inlining failed: ${e.message}\n`);
    calls_inlining_fails += 1;
  }
}

println("Results:");
println(
  JSON.stringify(
    { calls_total, calls_no_definition, calls_inlined, calls_inlining_fails },
    undefined,
    2
  )
);
\end{lstlisting}